\documentclass{article}
\usepackage[spanish]{babel}
\usepackage[round]{natbib}
\usepackage{authblk}
\usepackage[utf8]{inputenc} % allow utf-8 input
\usepackage[T1]{fontenc}    % use 8-bit T1 fonts
\usepackage[colorlinks=true,linkcolor=blue, citecolor=blue]{hyperref}%
     % hyperlinks
\usepackage{url}            % simple URL typesetting
\usepackage{booktabs}       % professional-quality tables
\usepackage{amsfonts}       % blackboard math symbols
\usepackage{nicefrac}       % compact symbols for 1/2, etc.
\usepackage{microtype}      % microtypography
\usepackage{lipsum}
\usepackage{multirow} 
\usepackage{pifont}% http://ctan.org/pkg/pifont
\usepackage{graphicx}
\usepackage{subcaption}
\newcommand{\cmark}{\ding{51}}%
\newcommand{\xmark}{\ding{55}}%
\title{Clasificación de textos con aprendizaje máquina (Programa 1)}


\author{Nombres de los miembros del equipo}

\affil{UNAM, FI Procesamiento del Lenguaje Natural 2025-2}


\begin{document}
\maketitle

\begin{abstract}
Un resumen tiene como objetivo conciso (menos de 300 palabras) presentar el problema, exponer los objetivos y destacar los principales descubrimientos.
\end{abstract}

\section{Introducción}
La introducción tiene como propósito explicar el problema, su relevancia, dificultad o importancia, evaluar los éxitos y fracasos de los métodos actuales en relación con el problema.


\section{Metodología}
En esta sección se detallan los enfoques utilizados para abordar el problema. 

\section{Experimentos y resultados}
En esta sección se presentan los resultados cuantitativos obtenidos y se incluye una evaluación cualitativa.

\section{Conclusiones}
En esta sección se resumen los principales hallazgos y lo que se ha aprendido. 





\bibliographystyle{agsm}
\bibliography{references} 


\end{document}
